\section{Task 3+5}
The algorithm we decided on works the following way:\\

\begin{itemize}
	\item Check if the cell that should get placed on exists.
	\begin{itemize}
		\item Case No: Return that the move is not valid.
	\end{itemize}
	\item Check if the player has a override stone if the move needs it.
	\begin{itemize}
		\item Case No: Return that the move is not valid.
	\end{itemize}
	\item for each(direction of the cell)
	\begin{itemize}
		\item while(next cell not an wall, the starting, an empty and a player cell)
		\begin{itemize}
			\item Get the new cell.
			\item Get the new direction.
		\end{itemize}
		\item if(next cell is from the player)
		\begin{itemize}
			\item Reverse the direction.
			\item while(next cell not the starting cell)
			\begin{itemize}
				\item Mark that the cell needs to get recoloured.
				\item Get the new cell.
				\item Get the new direction.
			\end{itemize}
		\end{itemize}
	\end{itemize}
	\item if(at least one cell was marked)
	\begin{itemize}
		\item Recolour all marked cells.
		\item switch(starting cell state)
		\begin{itemize}
			\item case player or expansion: Take away one override stone of the player.
			\item case inversion: Do the inversion.
			\item case choice: Switch the stones of the player with stones of the chosen other player.
			\item case bonus: Give the player a bomb or an override stone, depending on his choice.
			\item case empty: Nothing special to do here.
		\end{itemize}
		\item Recolour starting cell.
		\item Draw map.
		\item Return that the move was valid.
	\end{itemize}
	\item if(no cell was marked and starting stone is a expansion stone)
	\begin{itemize}
		\item Recolour starting cell.
		\item Take away one override stone of the player.
		\item Draw map.
		\item Return that the move was valid.
	\end{itemize}
	\item Return that the move is invalid.
\end{itemize}
