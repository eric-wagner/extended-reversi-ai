\documentclass[a4paper,12pt]{article}

\usepackage[a4paper]{geometry}

\usepackage[utf8]{inputenc}            % Use utf8 input encoding
%\usepackage[latin1]{inputenc}         % Use iso 8859-1 encoding
\usepackage[T1]{fontenc}               % T1 fonts (support for accents/diacritics)
\usepackage{lmodern}                   % font with proper T1 support and good glyph quality

\usepackage{listings}                  % for (code) listings
\usepackage{amsmath}                   % AMS math typesetting
\usepackage{titlesec}

\usepackage{hyperref}                  % better references for PDF

\usepackage{color}
\usepackage{listings}
\usepackage{tabularx}
\usepackage{array}
\usepackage{calc}
\usepackage[table]{xcolor}
\usepackage{graphicx}
\usepackage{verbatimbox}

\newcolumntype{z}[1] {
		@{{\centering \parbox[c]{\tabcolsep}{\rule{0pt}{#1 + 2\tabcolsep}}}}
		>{\centering\arraybackslash}
		m{#1} }

\lstset{ %
	language=C++,                % choose the language of the code
	basicstyle=\footnotesize,       % the size of the fonts that are used for the code
	numbers=left,                   % where to put the line-numbers
	numberstyle=\footnotesize,      % the size of the fonts that are used for the line-numbers
	stepnumber=1,                   % the step between two line-numbers. If it is 1 each line will be numbered
	numbersep=5pt,                  % how far the line-numbers are from the code
	backgroundcolor=\color{white},  % choose the background color. You must add \usepackage{color}
	showspaces=false,               % show spaces adding particular underscores
	showstringspaces=false,         % underline spaces within strings
	showtabs=false,                 % show tabs within strings adding particular underscores
	frame=single,           % adds a frame around the code
	tabsize=2,          % sets default tabsize to 2 spaces
	captionpos=b,           % sets the caption-position to bottom
	breaklines=true,        % sets automatic line breaking
	breakatwhitespace=false,    % sets if automatic breaks should only happen at whitespace
	escapeinside={\%*}{*)}          % if you want to add a comment within your code
}

\titleformat{\section}{\LARGE\bfseries}% hide redundant number
            {}{0pt}{}



\begin{document}


\begin{center}
	\rule{\textwidth}{0.1pt}\\[1cm]
	
	\Large Softwarepraktikum SS 2016\\\bf Assignment 1 
\end{center}


\begin{center}

	\rule{\textwidth}{0.1pt}\\[0.5cm]

	{\Large Group - 3\\[5mm]} 

	\begin{tabular}{lll}
		
		Eric Wagner & 344137 & eric.michel.wagner@rwth-aachen.de \\

		Stefan Rehbold & 344768 & Stefan.Rehbold@rwth-aachen.de \\

		Sonja Skiba & 331588 & sonja.skiba@rwth-aachen.de \\

	\end{tabular}\\[0.5cm]

	\rule{\textwidth}{0.1pt}\\[1cm]

\end{center}

% Uncomment next two lines for table of contents
%\newpage
%\tableofcontents

\section{Task 1}

For the first task of this weeks assignment we did write the code to connect our AI to a game server. Therefore we had to write the code that connects to an ip address and the port on which the server is listening. \\

During the setup phase a lot can go wrong, like not having the right ip or port or trying to connect to an offline server. To avoid any complications, we did try to catch every possible error and output a meaningful error message. After an unsuccessful connection attempt, our program will end itself as a controlled ending is better running under an undefined behaviour. And that way we can make sure that no memory leaks or open ports remain after the end of the program.\\

After the connection has been set up successfully we retrieve our player number and the map. In the following part of the code, we try to separate the receiving of a message and the control of the game flow as much as possible. To archive this, we did create a \textbf{listenToSever()} method which will wait for a message of the server and set a variable to whatever message has been sent. After that it will decode and save the transferred data.\\
In our main method we will take care of the game flow. First we ask which message was send last and then we execute the according code by using the data saved by the \textbf{listenToServer()} method. After a message has been processed, the AI is ready to get receive the next message. This will continue until the end of the game is announced by the server or an error accrued during the data transfer, to which our program would react with an controlled shutdown.

\section{Task 2}

For this task, we implemented the function 
\begin{lstlisting}[frame=none, numbers=none]
int32_t alphabeta(Map* map, int turn, int depth, bool isPlayingPhase, int alpha, int beta);
\end{lstlisting}
which is executing Alpha-Beta pruning, which prunes the “irrelevant” branches of the search tree. 
Most of the variables are the same as those of the MiniMax algorithm. The cut-off value is a given \emph{depth}, which determines how many move-combinations are going to be evaluated as a upper bound. The \emph{turn} (which players turn it is to make a move) needs to be known, as the node value shall be maximised for the own players turn, or else minimised. Also we need to know, which phase the game currently is in, which can be playing or bombing phase, so that the corresponding evaluation function is used. Newly added variables are the \emph{alpha} and \emph{beta} values which describe the worst case scenario for the player. \\
A node represents a game state and multiple nodes may have the same game state. The children of a node are nodes, that result from their parents game state by a move from the current drawing player. That way a player's move is represented by the branches leaving a specific node. The board will be evaluated at some of the leaves of the game tree. Leaves will not get evaluated if a sister or brother leaf was already evaluated with a bigger value then \emph{alpha}. \emph{alpha} is called the first time with the minimal value possible and \emph{beta} with the maximum value possible of the variable type, in this case int.
\\\\
If the algorithms depth is 0, a leaf is reached. The state will then be analysed and the evaluation value returned. If the depth is yet 1 or greater, the function iterates through the map and goes deeper. In a maximizing case, the evaluated score gets compared with \emph{alpha} and becomes the new \emph{alpha} if it is higher. In a minimizing case, the evaluated score gets compared with \emph{beta} and becomes the new \emph{beta} if it is lower. If \emph{alpha} is higher or equal to \emph{beta} the rest of the possible moves of the current state of the board won't get evaluated. Otherwise for every possible move from the current state of the board by calling 
\begin{lstlisting}[frame=none, numbers=none]
alphabeta(&mapCopy,(turn%numberOfPlayers)+1 ,depth-1, true, alpha, beta);
alphabeta(&mapCopy,(turn%numberOfPlayers)+1 ,depth-1, false, alpha, beta);
\end{lstlisting}
The first gets called in the playing phase and the second one in the bombing phase. \\
\section{Task 3+5}
The algorithm we decided on works the following way:\\

\begin{itemize}
	\item Check if the cell that should get placed on exists.
	\begin{itemize}
		\item Case No: Return that the move is not valid.
	\end{itemize}
	\item Check if the player has a override stone if the move needs it.
	\begin{itemize}
		\item Case No: Return that the move is not valid.
	\end{itemize}
	\item for each(direction of the cell)
	\begin{itemize}
		\item while(next cell not an wall, the starting, an empty and a player cell)
		\begin{itemize}
			\item Get the new cell.
			\item Get the new direction.
		\end{itemize}
		\item if(next cell is from the player)
		\begin{itemize}
			\item Reverse the direction.
			\item while(next cell not the starting cell)
			\begin{itemize}
				\item Mark that the cell needs to get recoloured.
				\item Get the new cell.
				\item Get the new direction.
			\end{itemize}
		\end{itemize}
	\end{itemize}
	\item if(at least one cell was marked)
	\begin{itemize}
		\item Recolour all marked cells.
		\item switch(starting cell state)
		\begin{itemize}
			\item case player or expansion: Take away one override stone of the player.
			\item case inversion: Do the inversion.
			\item case choice: Switch the stones of the player with stones of the chosen other player.
			\item case bonus: Give the player a bomb or an override stone, depending on his choice.
			\item case empty: Nothing special to do here.
		\end{itemize}
		\item Recolour starting cell.
		\item Draw map.
		\item Return that the move was valid.
	\end{itemize}
	\item if(no cell was marked and starting stone is a expansion stone)
	\begin{itemize}
		\item Recolour starting cell.
		\item Take away one override stone of the player.
		\item Draw map.
		\item Return that the move was valid.
	\end{itemize}
	\item Return that the move is invalid.
\end{itemize}

\section{Task 4}

The following aspects should be considered when rating the result of any given move.
\\\textbf{build phase rating function:}
\begin{itemize}
	\item number of own tiles, weighted by stability\textit{(sorted from best to worst)}: 
		\begin{itemize}
			\item stable tiles: tiles that can not be captured by the opponent at all
			\item semi-stable tiles: tiles that can be captured by the opponent, but not immediately in the next move/hard to capture in general) 
			\item unstable tiles: tiles that can be captured by the opponent in the next move
			\item high risk tiles: tiles that can be captured by the opponent in the next move and give them stable tiles
		\end{itemize}
	\item number of bombs captured in that move * effectiveness of bombs\textit{(=number of tiles a bomb hits)}
	\item number of own possible moves (maximize as far as possible)
	\item number of each opponents' possible moves (minimize as far as possible)
	\item minimized tile loss of player=((own playerindex) - (overall capturable inversion stones))
	\item number of available override stones (more valuable the closer it is to the end of the game/the less the opponents have left)
\end{itemize}
\textbf{bomb phase rating function}:
\begin{itemize}
	\item number of own tiles
	\item difference to the other players' number of tiles
\end{itemize}


\end{document}
