\documentclass[a4paper,12pt]{article}

\usepackage[a4paper]{geometry}

\usepackage[utf8]{inputenc}            % Use utf8 input encoding
%\usepackage[latin1]{inputenc}         % Use iso 8859-1 encoding
\usepackage[T1]{fontenc}               % T1 fonts (support for accents/diacritics)
\usepackage{lmodern}                   % font with proper T1 support and good glyph quality

\usepackage{listings}                  % for (code) listings
\usepackage{amsmath}                   % AMS math typesetting
\usepackage{titlesec}

\usepackage{hyperref}                  % better references for PDF

\usepackage{color}
\usepackage{listings}
\usepackage{tabularx}
\usepackage{array}
\usepackage{calc}
\usepackage[table]{xcolor}
\usepackage{graphicx}
\usepackage{verbatimbox}
\usepackage{pifont}
\usepackage{tikz}
\usepackage{pgfplots}

\newcolumntype{z}[1] {
		@{{\centering \parbox[c]{\tabcolsep}{\rule{0pt}{#1 + 2\tabcolsep}}}}
		>{\centering\arraybackslash}
		m{#1} }

\lstset{ %
	language=C++,                % choose the language of the code
	basicstyle=\footnotesize,       % the size of the fonts that are used for the code
	numbers=left,                   % where to put the line-numbers
	numberstyle=\footnotesize,      % the size of the fonts that are used for the line-numbers
	stepnumber=1,                   % the step between two line-numbers. If it is 1 each line will be numbered
	numbersep=5pt,                  % how far the line-numbers are from the code
	backgroundcolor=\color{white},  % choose the background color. You must add \usepackage{color}
	showspaces=false,               % show spaces adding particular underscores
	showstringspaces=false,         % underline spaces within strings
	showtabs=false,                 % show tabs within strings adding particular underscores
	frame=single,           % adds a frame around the code
	tabsize=2,          % sets default tabsize to 2 spaces
	captionpos=b,           % sets the caption-position to bottom
	breaklines=true,        % sets automatic line breaking
	breakatwhitespace=false,    % sets if automatic breaks should only happen at whitespace
	escapeinside={\%*}{*)}          % if you want to add a comment within your code
}

\titleformat{\section}{\LARGE\bfseries}% hide redundant number
            {}{0pt}{}



\begin{document}


\begin{center}
	\rule{\textwidth}{0.1pt}\\[1cm]
	
	\Large Softwarepraktikum SS 2016\\\bf Assignment 5
\end{center}


\begin{center}

	\rule{\textwidth}{0.1pt}\\[0.5cm]

	{\Large Group - 3\\[5mm]} 

	\begin{tabular}{lll}
		
		Eric Wagner & 344137 & eric.michel.wagner@rwth-aachen.de \\

		Stefan Rehbold & 344768 & Stefan.Rehbold@rwth-aachen.de \\

		Sonja Skiba & 331588 & sonja.skiba@rwth-aachen.de \\

	\end{tabular}\\[0.5cm]

	\rule{\textwidth}{0.1pt}\\[1cm]

\end{center}

% Uncomment next two lines for table of contents
%\newpage
%\tableofcontents

\section{Task 1 \\ Implementation of Aspiration Windows}
The first task of this biweekly assignment was to implement aspiration windows. As we have already implemented a way to select different algorithm, we did the same with aspirational windows to make to sure that we are able to disable them at any point in time.\\\\
The implementation of the algorithm is based upon alpha-beta pruning with move sorting enabled. The first iteration of iterative deepening is the same as previously with values of $-\infty$ for $alpha$ and $+\infty$ for $beta$.\\
We now use the score of the best move found and try to predict the a window between which we expect the score of the next move to be in. As we use a paranoid search strategy, we know that the score will most likely go down if one of our opponents will do the move that was not captured by the previous depth limit. So we set the $alpha$ value to the score of the current best move.\\ Otherwise, if we do the last move, our score should go up. Therefore, in these cases, we set the beta value to the current score.\\
Our first intuition to set the second value was to scale it up by a constant factor. This was not working at all, mainly because we can have a very low score but big addends(because of positive and negative factor cancelling out). So we increase/decrease the score be a fixed value depending on size of the map. Another problem was that we thought we could predict if the score goes down or not. This assumption was not true, most likely because of the way we evaluate the special stones.\\\\
Finally we used the formula:

$alpha = previousScore - x*getAmountOfCells()$

$beta = previousScore + x*getAmountOfCells()$\\
where $x$ was the variable we tried to find the optimal value for.\\\\
If we the value returned by the move search is not between the alpha and beta value we know that we failed, low or high. In that case we will try again with just one value changed, but by a bigger margin. If we fail high, we will increase beta. Otherwise we will decrease alpha.\\ If we still fail after those adjustments, we will use $-\infty$ and $+\infty$ again.\\\\
After trying a lot of values for $x$ we concluded that the aspirational windows do not help improve our search. The values at which we start pruning away some branches are very close to the ones where we fail to find a move at all. The range is sometimes lower than 1. This means that even if we manually set the $x$ value we see only small improvements for a very small range for $x$. Meanwhile the $x$ value we did find to work for different benchmarks ranged from 3 to 100.\\\\
We did not include graphics as there where no informations to be gained from them other than described above, and no nice way to display what we did. We will not use aspirational windows for the moment. Before the tournament we might try to active them again if our final evaluation functions seem more suitable.


\section{Task 2}

For this task, we implemented the function 
\begin{lstlisting}[frame=none, numbers=none]
int32_t alphabeta(Map* map, int turn, int depth, bool isPlayingPhase, int alpha, int beta);
\end{lstlisting}
which is executing Alpha-Beta pruning, which prunes the “irrelevant” branches of the search tree. 
Most of the variables are the same as those of the MiniMax algorithm. The cut-off value is a given \emph{depth}, which determines how many move-combinations are going to be evaluated as a upper bound. The \emph{turn} (which players turn it is to make a move) needs to be known, as the node value shall be maximised for the own players turn, or else minimised. Also we need to know, which phase the game currently is in, which can be playing or bombing phase, so that the corresponding evaluation function is used. Newly added variables are the \emph{alpha} and \emph{beta} values which describe the worst case scenario for the player. \\
A node represents a game state and multiple nodes may have the same game state. The children of a node are nodes, that result from their parents game state by a move from the current drawing player. That way a player's move is represented by the branches leaving a specific node. The board will be evaluated at some of the leaves of the game tree. Leaves will not get evaluated if a sister or brother leaf was already evaluated with a bigger value then \emph{alpha}. \emph{alpha} is called the first time with the minimal value possible and \emph{beta} with the maximum value possible of the variable type, in this case int.
\\\\
If the algorithms depth is 0, a leaf is reached. The state will then be analysed and the evaluation value returned. If the depth is yet 1 or greater, the function iterates through the map and goes deeper. In a maximizing case, the evaluated score gets compared with \emph{alpha} and becomes the new \emph{alpha} if it is higher. In a minimizing case, the evaluated score gets compared with \emph{beta} and becomes the new \emph{beta} if it is lower. If \emph{alpha} is higher or equal to \emph{beta} the rest of the possible moves of the current state of the board won't get evaluated. Otherwise for every possible move from the current state of the board by calling 
\begin{lstlisting}[frame=none, numbers=none]
alphabeta(&mapCopy,(turn%numberOfPlayers)+1 ,depth-1, true, alpha, beta);
alphabeta(&mapCopy,(turn%numberOfPlayers)+1 ,depth-1, false, alpha, beta);
\end{lstlisting}
The first gets called in the playing phase and the second one in the bombing phase. \\

\end{document}
